\chapter{Methodology}
\label{chapter:methods}

This chapter will describe in detail the research goals of the thesis project. Besides, I will also outline the scope of the thesis work.

\section{Research goals}

As has been shown in the previous chapters, a lot is known about classification
of LCL problems on trees. In particular, plenty of research has been done
on the topic of automated classification of LCL problems. However, the
results of the research are scattered across numerous papers. Moreover,
all of the meta-algorithms described use different representation for
the LCL problems that they accept as an input.
The issue gets even more complicated by the fact that the existing practical implementations
of the meta-algorithms are written in different programming languages, accept input in
different forms,
use different internal representation of a problem and produce output in different
formats.

In addition to this, researchers would
have to download and install multiple software packages on their machines and learn
the formalisms for representing LCL problems for each of the meta-algorithm
representations. Most importantly, since each of the meta-algorithms' implementations
is capable of classifying only a relatively narrow subfamily of LCL problems, the
researchers would have to constantly keep in mind the information about what families
of problems can be classified by what meta-algorithm.
All this makes it difficult and impractical for members of
the research community
to use the meta-algoithms in their everyday work.

Therefore, the goal of this thesis is a software tool that would store most -- if not all -- of
the so far existing knowledge on complexity of LCL problems in trees.
The tool would, in a sense, encapsulate all of the capabilities
of the existing meta-algorithms.
The tool would also allow for queries regarding a specific problem or a group of
problems. As an example, we can imagine the following two types of queries:
\begin{itemize}
  \item Is the complexity of a problem $X$ already known (based on the existing meta-algorithms
  and/or accumulated individual complexity results)? If so, what is the complexity in both
  deterministic and randomised settings?
  \item Return all problems on \emph{binary rooted trees} that have complexity of
  $\Theta(\log^* n)$.
\end{itemize}
In addition to this, the tool’s output would include references to
the sources
(published papers, certain theorems, ad-hoc implemented scripts, etc.)
which the returned results are based on.

All this would make it more accessible for members of the research
community to use the powers of the existing meta-algorithms.
This, we believe, will be highly useful for the distributed
algorithms researchers, as it would allow them to get access
to existing information about complexity of LCL problems on trees
in a matter of seconds.

\section{Scope}

When selecting the scope of the thesis, it is necessary
to decide on the limits
in terms of the properties of the software. I present the limiting properties
as a list of questions.

\begin{enumerate}
  \item What graph families should my tool be able to work with?
  \item What meta-algorithms are to be used as part of the solution?
  \item Are datasets of individual complexity results to be used
  in addition to the meta-algorithms? If so, which of the datasets should be used?
  \item How general should the query be, and what parameters the query should include?
\end{enumerate}

To answer these questions, I utilize the existing literature on the subject, which was partly presented in Chapters~\ref{chapter:background} and~\ref{chapter:environment}. As was already explained earlier, problems on general graphs are not decidable. At the same time, all of the results listed in Chapter~\ref{chapter:environment} can be applied to problems on trees. Thus,
I have decided to limit ourselves to trees as an answer to the first question in the list.

At the moment of writing the thesis, the number of meta-algorithms is
quite small. Indeed, all of the meta-algorithms that the author is aware of were listed and explained in Chapter~\ref{chapter:environment}. Therefore, there are only five meta-algorithms to use: the round elimination technique, the automata-theoretic lens classifier, classifier for binary labeling problems on unrooted trees, classifier for ternary labelling problems on unrooted trees, and classifier for problems on rooted trees. Moreover, for each of these classifiers, there exists an practical implementation of the meta-algorithm, even if in a limited form (e.g.\ an implementation of the classifier for problems on rooted trees, currently only supports problems on \emph{binary} rooted trees). Thus, I have decided to use all of the five listed meta-algorithm as part of the solution.

There are only two LCL problem datasets the author is aware of. One of them is the dataset of binary and ternary labeling problems on binary rooted trees, which is a
part of the Python package for classifying binary rooted trees~\cite{Tereshchenko2020brt}. The other one is the dataset of binary and ternary labeling problems on
unrooted $(2, 2)$-biregular and $(3, 2)$-biregular trees, which is a part of the TLP classifier Python package~\cite{Rocher2020clas}. Since the latter dataseet is already incorporated in the TLP classifier,
the former dataset is the only one that I eventually decided to use
in addition to the aforementioned meta-algorithms.

Finally, I will attempt to answer the last question.
This is perhaps the most difficult one from the list,
as the research needs of different people involved in
the distributed algorthms research might differ a lot, and thus
the parameters of the query that might prove to be useful
will vary from case to case too. It was thus decided to first,
based on some assumptions and general knowledge of the
needs of the research community, to come up with some form of the
query -- even if somewhat arbitrary -- and release an early version
of the software as soon as possible to allow people on the
research community to use the tool. This has enabled us to
collect feedback about the implementation in general and about the
query functionality in particular, and to reiterate the structure of
the query based on the feedback received. As the result, it was
decided that the query functionality should be able to perform
filtering based on the following properties of a problem:

\begin{itemize}
  \item Randomised complexity.
  \item Deterministic complexity.
  \item The graph family of problems (e.g.\ whether the problem in on
  rooted or unrooted trees, whether the underlying graph is a tree
  or a cycle, degree of the grah if a graph is regular, etc.).
  \item The number of labels allowed in the solution of the problems.
  \item Restrictions on what outputs can nodes produce (e.g.\ the
  produced output must always iniclude 2-coloring of a graph,
  or all passive nodes in a graph must output the same
  label on all incident edges, etc.).
  \item Possibilty to query for only the \emph{smallest} problem that
  satisfies other query parameters. Noice that the \emph{smallest} term here is
  defined rather ambiguously and is open for interretation while implementing. Roughly speaking, it means a problem with the smallest
  number of valid outputs.
  \item Possibilty to query for only the \emph{largest} problem. Here again
  the \emph{largest} term is a rather ambiguous term and is used in a simlar sense
  as explained above.
  \item Possibility to query for the smallest not-yet-classified problem.
\end{itemize}
