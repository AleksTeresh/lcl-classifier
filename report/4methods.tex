\chapter{Methodology}
\label{chapter:methods}

This chapter will describe in detail the research goals the present work
is trying to achieve. Besides, we will also describe the aproach we decided
to take to achieve the goals and provide justifications for why this particular
approach has been chosen. Finally, we will outline the scope of the work.

\section{Research goals}

As has been shown in the previous chapters, a lot is known about classificaiton
of LCL problems on trees. In particular, plenty of research has been done
on the topic of automated classification of LCL problems. However, the
results of the research are scattered across numerous papers. Moreover,
all of the meta-algorithms described use different representation for LCL
problems that they accept as an input.
The issue gets even more complicated by the fact that the existing practical representations
of the meta-algorithms are writted in different programming languages, accept input in
different forms,
use different internal representation of a problem and produce output in different
formats.

The goal of this thesis is a software tool that would store most -- if not all -- of
the so far existing knowledge on complexity of LCL problems in trees.
The tool would also allow for queries regarding a specific problem or a group of
problems. As an example, we image the following two types of queries:
\begin{itemize}
  \item Is complexity of a problem $X$ already known (based on the existing meta-algorithms
  and/or accumulated individual complexity results)? If so, what is the complexity in both
  deterministic and randomised settings?
  \item return all problems on \emph{binary rooted trees} that have complexity of
  $\Theta(\log* n)$.
\end{itemize}
In addition to this, the tool’s output would include references to sources
(published papers, certain theorems, ad-hoc implemented scripts, etc.)
which the returned results are based on.

\section{Approach}

\section{Scope}
