\chapter{Introduction}
\label{chapter:intro}

Distributed computing is used in a wide variaty of areas
such as telecommunication, data processing, process control in real-time systems, etc~\cite{Lynch1996}. Similar to
how traditional algorithms are studying with a classical
centralized computational entity in mind, distributed
algorithms are developed to be used in distributed
systems that are becoming more and more ubiquitous in the
field of technology~\cite{Attiya2004}. Such distributed
systems consist of many computational units each
performing local computation and communicating
with other such units close to it.

The field of distributed algorithms has several
decades of research history and has had grown
in width significantly since its inception in 1980s.
One specific
area of the field that has received a lot of
attention in the recent years from scientific
community is the study of \emph{locally verifiable}
problems.
These are the kinds of problems, where, roughly speaking,
once an execution of an algorithm has ended, each
computation unit in the distributed system can just
collect information from other compputational units
close to it to ensure that the executed algorithm has succeeded. Problems of this kind are particularly
interesting because they have obvious relevance
in cases where the size of the network is extremely
large, and thus each computational unit
cannot afford to check every other unit in the system.

\section{Problem statement}



\section{Structure of the Thesis}
\label{section:structure} 
