\chapter{Introduction}
\label{chapter:intro}

Distributed computing is used in a wide variaty of areas
such as telecommunication, data processing, process control in real-time systems, etc~\cite{Lynch1996}. Similar to
how traditional algorithms are studying with a classical
centralized computational entity in mind, distributed
algorithms are developed to be used in distributed
systems that are becoming more and more ubiquitous in the
field of technology~\cite{Attiya2004}. Such distributed
systems consist of many computational units each
performing local computation and communicating
with other such units close to it.

The field of distributed algorithms has several
decades of research history and has had grown
in width significantly since its inception in 1980s.
One specific
area of the field that has received a lot of
attention in the recent years from scientific
community is the study of \emph{locally verifiable}
problems.
These are the kinds of problems, where, roughly speaking,
once an execution of an algorithm has ended, each
computation unit in the distributed system can just
collect information from other compputational units
close to it to ensure that the executed algorithm has succeeded. Problems of this kind are particularly
interesting because they have obvious relevance
in cases where the size of the network is extremely
large, and thus each computational unit
cannot afford to check every other unit in the system.

As the field has been studied, more and more knowledge
about such problems has been accumulated by the
research community. In particular, as is common
in the field of algorithm design, much of the
research has been focused on complexity of
problems in distributed computing. Apart from
individual compplexity results, the research
community has invented several "meta-algorithms".
These are centralized algorithms that
automatically determine
computational complexity of a provided
problem.

\section{Problem statement}



\section{Structure of the thesis}
\label{section:structure} 

Chapter~\ref{chapter:background} provides a theoretical
background necessary for understanding the contents of the
following chapters. The chapter opens with
a brief but informative overview of graph-theoretic
concepts. Then, the chapter provides basic
theoretical background related to distributed computing.
Moreover, the chapter explains notions of
locally checkable problems, decidability, as well as
covers recent developments in the aread of
automatic classification in distributed algorithms.

Chapter~\ref{chapter:environment} describes
meta-algorithms that are used in our final
solution. The meta-algorithms take a description
of a locally verifiable problem as an input and
as an output produce some information about the
problem's compplexity.

Chapter~\ref{chapter:methods} states our research goals
and defines the scope of the thesis project.
Chapter~\ref{chapter:implementation}
provides an overview of an implementation of
our solution. Chapter~\ref{chapter:evaluation}
evaluates the outcomes of the implementation
against the earlier stated research goals. It
also points out possible directions for
future development. Finally, Chapter~\ref{chapter:conclusions}
summarizes the entirety of the work.

