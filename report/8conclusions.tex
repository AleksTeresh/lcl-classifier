\chapter{Conclusions}
\label{chapter:conclusions}

The study of distributed algorithms has become
one of the central topics in theoretical computer science
over the last twenty years. Many significant results in the area
have been achieved, especially when it comes to topics related
to computational complexity. Some of these results
try to move the field into a direction where deciding the
computational complexity of a distributed problem becomes
increasingly automated. A long-term goal behind this is
to show theoretically -- and often practically via an
implementation -- that deciding the complexity of a certain class
of problems can be done in a mechanical manner relying
on systematic algorithms instead of ad-hoc proof techniques
by humans.

In this thesis, I have shown that, although the work on
decidability of LCL problems has seen a surge over the last
several years in terms of research output, the results
failed to adopt a unified framework or formalism. This
prevents the research community from utilizing the
results in their everyday work, even though practical
implementations for many of the theoretical results have been
made available. Therefore, as the goal of this project, I
envisioned a solution that would unify the individual
implementations behind a common interface. Moreover, to make
the system more useful to the research community, I had decided
to implement a Web interface for the tool.

Although I have not reached the rather ambitious initially formulated
research goal of encapsulating ``most -- if not
all -- of
so far existing knowledge on the complexity of LCL
problems on trees'', I did succeed in creating a useful
tool that unifies most of the existing implementations that
provide some possibility of computer-aided classification of LCL
problems. Moreover, the tool has been designed and implemented in
such a way that it can be easily extended with more meta-algorithms
and other decidability implementations if such were to appear.
Some of the details of the implementation are described
in Chapter~\ref{chapter:implementation}.

Finally, the groundwork done will allow for numerous
extensions and improvements to the tool. I described
some of the opportunities for future development in
Chapter~\ref{chapter:evaluation}. Apart from
integrating more meta-algorithms and problem datasets into the
solution, a lot can be improved regarding what data
is stored and displayed to the user
(providing the information about the exact constant complexities of
LCL problems is just one example of this). Besides,
possibilities related to populating the database with more
families of pre-classified problems are virtually limitless.
Furthermore, getting feedback from the research community and
developing the query functionality to become even more
relevant is yet another way to continue the current
project.
