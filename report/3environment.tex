\chapter{Meta-algorithms used in the solution}
\label{chapter:environment}

This chapter will describe in detail all the meta-algorithms that are
used in the final solution as subroutines.
Chapter 5 will then
describe how the final solution was built on top of the meta-algorithms
described in this chapter.

\section{Round Elimination}

This section explains in detail Round Elimination
technique introduced in Chapter 2, as well as
implementation of Round Elimination as a computer
program written in Rust \cite{Brandt2019, Olivetti2020}.
Besides, we will explain the relevance of the technique
and the software implementation for our problem.

As already mentioned in the previous chapter, round elimination
is a technique that, given an LCL problem $\Pi_0$ as an input, produces
another LCL problem $\Pi_1$ which can be solved exactly one round
faster. For round elimination to work, the input should comply
to two constraints: $\Pi_0$ has to be a problem on a $(\delta, \beta)$-
biregular graph, and the number of rounds in which $\Pi_0$ can be solved
should not be "too large". Indeed, the last constraint is a rather curious one
since we rarely know round-complexity of an input problem when using
round elimination. However, we can nevertheless use this rather inconvinient constraint
to our advantage, which will be demonstrated below.

Furthermore, when applying round elimination we talk about "active" and
"passive" nodes. Since $(\delta, \beta)$-bipartition of an input problem is
already given (see our assumption above), nodes of degree $\delta$ are assumed
to be "active" and nodes of degree $\beta$ are assumed to be passive. For a problem
to be a valid input to round elimination, it has to be reformulated as a problem on
a $(\delta, \beta)$-biregular graph where only one partition of nodes produces
some output while the other partition does not produce any output but instead
checks that their radius-1 neighbourhood's outputs comply to previously
specified restrictions.

In order to demonstrate the technique, we will have as our running
examples two canonical problems: weak 3-labelling and sinkless orientation.

\textbf{Weak 3-labeling} in this context is a problem on $delta$-regular trees where
each node labels its incident edges in such a way that no node $v$ in the graph
has all its incident edges labelled with the same label. Besides if two
nodes $v$ and $u$ are incident to the same edge $e = \{v, u\}$, such
edge $e$ has to be labelled with the same label from both "sides". In other words,
two neighbouring nodes cannot output different labels on the same edge.
It is easy to see although the initial problem is specified for a regular
tree, we can obtain an equivalent problem for a $(\delta, 2)$-biregular tree
by replacing edges with nodes as described in subsection ~\ref{subsection:biregular-trees}.
Further, if we assume that all nodes of degree $\delta$ are active and all nodes of 
degree 2 are passive and that nodes of degree $\delta$ cannot have their incident
edges labelled all with the same label, and nodes of degree 2 must have both of their
incident edges labelled with the same label, we obtain an LCL problem equivalent to
weak 3-labelling but in a formalism suitable for round elimination. To formalise, we
can define the problem as following:

$$\Sigma = \{A, B, C\}$$
$$A = \{ \{A, B, C\}, \{A, A, B~or~C\}, \{B, B, A~or~C\}, \{C, C, A~or~B\} \}$$
$$P = \{ \{A, A\}, \{B, B\}, \{C, C\}\}$$
where $\Sigma$ is an allowed alphabet, $A$ is a set of \emph{configuration}
allowed for active nodes and $P$ is a set of configuration allowed for
passive nodes. Here we assumed that $\delta = 3$. Notice that each
configuration is a set of labels that are allowed to be on edges 
incident to any active/passive node $v$. Notice also that the order of
the labels in a single configuration does not matter. Finally, in this particular problem
and in all problems in the context of round elimination, we usually do not care about
leaf nodes. That is leaf nodes are unrestrained and are fine with any 
configuration.  Next, we will
describe another problem using the same formalism.

\textbf{Sinkless orientation} is a problem on a $\delta$-regular tree
where each non-leaf node has outdegree of atleast 1. Similar to the case of weak 3-labelling,
we can replace each edge with a node of degree 2 to get to $(\delta, 2)-biregular$ tree with
nodes of degree $\delta$ being active and nodes of degree 2 being passive.
Then we can define the problem formally as follows:

$$\Sigma = \{I, O\}$$
$$A = \{ \{O, I~or~O, I~or~O\} \}$$
$$P = \{ \{I, I~or~O, I~or~O\}\}$$


\section{Automata-theoretic lens classifier}

\section{Classification of binary labelling problems}

\section{Classification of ternary labelling problems}

\section{Classification of problems on rooted trees}

<Section on yet-to-be published paper>

