\chapter{Theoretical background}
\label{chapter:background}

In this chapter, we introduce the necessary theoretical background as well as
explain the reasons why the current work is important given the latest
developments in the theory of distributed computing.

\section{Graphs}

This section will outline the necessary theoretical background on graphs. 
Most of the notation related graph theoretic concepts in the current
section and in the rest of the thesis will follow that of
a recent introductory textbook on distributed algorithms~\cite{Hirvonen2020},
unless specified otherwise.

A \emph{graph} is a pair $G = (V, E)$, where $V$ denotes the set of all \emph{vertices}
and $E$ denotes a set of all \emph{edges}. Each edge in $E$ is represented as a set of
2 nodes i.e. the nodes the given edge is connecting e.g. $e = \{v, u\}$ such that 
$v \in V$ and $u \in V$.

When talking about the \emph{size of a graph}, or \emph{cardinality} of the graph, we refer
to the number of nodes in the graph. Or in other words, the size of a graph $G$
is always $|V|$.

Among other things, edges can be categorised into \emph{directed} and \emph{undirected}.
The latter ones simply connect a pair of nodes in a graph, while the latter
ones also contain extra information about which of the 2 connected nodes is
a "source" and which one is a "destination". Informally, such a directed edge
can be visualised as an error that starts in a "source" node and ends in a 
"destination" node.  Similarly, we talk about \emph{undirected graphs} - that is graphs
where all edges are undirected, and \emph{directed graphs} - such graphs where all
edges are directed. Formally, while an undirected edge $e$ is defined as a
set of 2 nodes $e = \{v, u\}$, a directed edge $e$ is defined as a set tuple
of 2 nodes $e = (v, u)$. Recall that in the case of a tuple, the order of its
elements does matter, hence, the tuple allows us to encode the direction of
the edge. Unless mentioned otherwise, we assume that an edge is directed
from the first element of the tuple to the second. Also note that in this
thesis, unless specified explicitly, all graphs are assumed to be undirected.

If two nodes are connected by an edge, we call such nodes \emph{neighbours}. We
also say that such nodes are \emph{adjacent} to each other. Moreover, in an undirected graph
if there are two edges $e_1$ and $e_2$ such that $e_1 \neq e_2$ and $e_1 \cap e_2 \neq \emptyset$
such edges are said to be \emph{adjacent} to each other. Similarly, we can talk about adjacent 
edges in a directed graph. In that case, two edges $e_1$ and $e_2$ are adjacent if
$e_1 \neq e_2$ and either first or second element of tuple $e_1$ contains the same
element as either first or second element of tuple $e_2$.

Besides we often talk about edges and nodes being \emph{incident} to each other.
In an undirected graph, an edge $e$ is incident to a node $v$ if and only if $v \in e$. In this
case, we also say that the node $v$ is incident to the edge $e$. Similarly, in a
directed graph, an edge $e$ is incident to a node $v$ if and only if $v$ is the
first or the second element of the tuple $e$. Again, in the case described in the
previous sentence, the node $v$ is said to be incident to the edge $e$.

A \emph{simple graph} is an undirected graph where no 2 nodes are connected by more
than one edge and no edge starts and ends at the same node. In other words, there are
no multiple edges or self-loops in a simple graph.

A \emph{degree} of a node $v \in V$ in a graph $G = (V, E)$ is the number of all
edges incident to $v$. That is it is the number of edges such that for an edge
$e$, $v \in e$ if $e$ is an undirect edge, or $v$ is the first or second element of
tuple $e$ if $e$ is a directed edge. Moreover, for directed graphs, we often talk
about \emph{indegrees} and \emph{outdegrees}. Indegree of a node $v$ is the number
of edges incident to $v$ directed towards the node, while outdegree of a node $v$ is the 
number of edges incident to $v$ directed outwards from the node. Finally, we are 
often interested in a maximum degree of a graph $G$, that is the maximum value of
degrees of all nodes belonging to the graph $G$. We denote such maximum degree as
$\Delta$

A \emph{walk} in an undirected graph $G = (V, E)$ is a sequence $w$ of a form
$w = (v_0, e_1, v_1, e_2, ..., e_l, v_l)$,
where $v_i \in V$ and $e_i \in E$, and $e_i = {v_{i-1}, v_i}$ for all $i$. A walk from some
node $v$ to some other node $u$ is then a walk $w$ such that its first element i.e. $v_0 = v$
and its last element i.e. $v_l = u$. Having defined a walk, we can talk about a \emph{
connected graph}. A connected graph, when talking about undirected
graphs, is a graph $G = (V, E)$ in which for any pair of nodes $v$ and $u$ such that
$v \neq u$ there is a walk from $v$ to $u$.

An \emph{isomorphism} between two graphs $G_1$ and $G_2$ is is a function $f$ such that $f$ is a 
bijection, and $f$ maps a vertex of the graph $G_1$ to a vertex of the graph $G_2$, and 
an edge between some nodes $v$ and $u$ exists in the graph $G_1$ if and only if
an edge between nodes $f(v)$ and $f(u)$ exists in the graph $G_2$. From this,
it is rather easy to see that if an isomorphism exists from $G_1$ to $G_2$, then
also an isomorphism exists from $G_2$ to $G_1$. If there exists an
isomorphism between some two graphs, we say that such graphs are isomorphic
(to each other).

A \emph{radius-x neighbourhood} of a node $v$ in a graph $G = (V, E)$ is a set of
all such nodes $u \in V$ that there exists a walk between $v$ and $u$ and 
the shortest walk from $v$ to $u$ is at most $x$. Note that it is possible that
$v = u$, in which case the shortest walk between $v$ and $u$ is 0.

A diameter of a graph (denoted as $diam(G)$) is the length of the longest walk
in a list $W$, where $W$ is a list of the shortest walks - one between each pair of nodes
$v$ and $u$ in a graph $G = (V, E)$ such that $v \in V$ and $u \in V$. If a graph
$G$ is nont connected and therefore there is a pair of nodes $v$ and $u$ that do not
have a walk between them, we say that a diameter of such a graph $G$ is infinity. That
is $diam(G) = \inf$.

\subsection{Several important graph families}

This subsection will introduce some of the graph families that will be necessary
for understanding the content of the thesis. First, we'll define paths and cycles.
Then, we'll briefly introduce trees and explain the difference between rooted
and unrooted trees.

A \emph{path} graph is a sequence of nodes that are joined together by edges and
that have the following properties:

\begin{enumerate}
  \item The whole graph is connected. That is there is a walk from any node of
  the path to any other node of the path.

  \item The graph consists of at least 2 nodes.

  \item Exactly 2 nodes have a degree 1 and all the other nodes have a degree 2.
\end{enumerate}

A \emph{cycle} is a connected graph where each node has a degree 2.

A \emph{tree} is an undirected acyclic connected graph. A \emph{rooted tree}
has a single \emph{root vertex}, and thus some nodes have a \emph{parent} node
(some because e.g. a root node never has a parent) and one or more \emph{children}
nodes. For a node $v$, a node $u$ is its parent if and only if $v$ and $u$ are
neighbour nodes and the shortest walk from the root to $u$ is shorter by exactly
1 compared to the shortest walk from the root to $v$. Node $v$ is a child of node
$u$ if and only if $u$ is a parent of node $v$. Finally, in the context of rooted trees,
a node $l$ that has no children is referred to as a \emph{leaf} node.
On the other hand, an \emph{unrooted tree} has no single root, and therefore the notion of parent
or child is not defined. Instead, we talk about neighbours of some node $v$. However, we 
still use the notion of leaves to denote nodes of degree 1.

\section{Distributed computing}

Now that we have introduced some of the key graph theoretic concepts, we will
next outline some of the foundations of distributed computing. We'll start with
a short historical node about the field of distributed computing. Then, we'll
explain some of the aspects of the model of computation we're concerned with. Finally, 
we'll give some formal definitions to the model of computation that will be our
major concern throughout the rest of the thesis.

\subsection{General background}

The field of distributed computing studies computation in distributed systems, which
have become ubiquitous in the modern world, being especially prevalent
in the area of technology~\cite{Attiya2004}.
A distributed system consists of a number of relatively independent
computing modules, which usually need to cooperate in order to
fulfill a computational task the system has been given. Usually,
each computing module in such a system only has a part of the whole
input and is required to produce only a part of the whole output.
This is in contrast with a centralised system, in which there 
exists an all-knowing entity taking all of the input, performing all
of the computation and producing the whole of a result as its output.
Due to its modularised and parallel nature, distributed computing
has many applications in communication, computation, the Internet,
but also in biology and sociology~\cite{Wattenhofer2016}.

The field of distributed computing appeared already in late 1980s, with several
prominent papers exploring potentialities of computation performed
by multiple interconnected processing units~\cite{Cole1986, Linial1987, Naor1991}.
The first major step in the field happened in 1987, when Linial formalised
some of the principles of one variant of a distributed system.
This model of computation is currently known as Linial's or \emph{LOCAL model}~\cite{Linial1987}.

\subsection{Model of computation}

Here, we will describe the model of computation that we are going to
assume throughout the rest of the thesis. Note that there is a number
of different models, which are also widely studied in the research
community.

First, as it was already stated, unlike in the case of centralised
computation, we are concerned with multiple interconnected computing entities
that together form a graph. Each such an entity is referred to as a vertex or
a node in a graph. Connections between the vertices are referred to as edges.
The entities only can transfer information along the edges and in no other way
between each other. Unless specified otherwise, the graphs are assumed to be
simple graphs, and all edges are assumed to be undirected. Moreover,
communication along the edges can simultaneously happen in both direction.

Apart from sending infromation to each other, nodes can also
do local computation based on the local information each node possesses. One thing
to note that differs significantly from some of the centralised models of computation
is that each vertex in a graph has arbitrarily huge, but finite, storage and computation resources.
Informally, this means that anything that can be computed in a centralised setting
in a finite amount of time,
can also be computed locally by a node in an arbitrarily small amount of time.
It is also worth noticing that in the model of computation described here,
every node of a graph executed the same algorithm. Nevertheless, the
algorithm might lead to different commands being executed by different nodes
if, for example, nodes' unique identifiers or initial local inputs are different.
Besides, behaviour might also differ if two nodes have somewhat different neighbourhoods
around themselves and therefore will receive potentially different
information during their communication rounds. Finally, at the beginning of
execution, each node knows only its own input (including its own unique identifier
if such has been provided as part of the input) and its own degree i.e. the number of
its neighbour nodes in the graph.

Furthermore, computation in a graph happens in synchronous rounds. That implies,
for example, that round $x+1$ is not started by any of the nodes before all of the
nodes have completed round $x$. Moreover, each round is divided into three stages:

\begin{enumerate}
\item Sending some information to some (or all) of its neighbours

\item Receiving information
from some (or all) of its neighbours

\item Performing some local computaiton based
on the information that has been stored by a node locally before and the information
received during the current round from its neighbours

\item Updating its local state i.e. replacing and adding data in its own local storage
\end{enumerate}

Each of these stages is also executed
completely synchronously by all nodes in a graph, meaning that e.g. no node starts processing
any of its data, before all other nodes have received data from their neighbours. The computation
of a graph is said to be finished when all of the nodes have outputted their final states
and halted. More formally, this means that there are a set of node states that are
considered as "halting states". Whenever a node switches its internal state to one of the 
halting state, it does not perform any of the actions from that point in time, or - to
be even more formal - it does not send any messages to its neighbours, ignores all of the 
messages sent to it and does not update its internal state in all of the subsequent rounds.
Similar to the requirement that local computation of each node on each round has to be finite
and needs to stop after a finite amount of time passed, all nodes are required to eventually
transition into one of the halting states. That is a valid distributed algorithm - in our model of
computing - cannot continue for an infinite number of rounds. Finally,
the complexity of a distributed algorithm is measured as number of such synchronous rounds
of computation before the algorithm ends i.e. before all nodes transition into one of 
the halting states. Notice that different from a centralised model of computation,
algorithms complexity (or in other words its running time) is not affected by the
amount of local computation on a single node.

Another important thing that has to be mentioned when describing the mode of computation
is that all of the operations withing each individual node as well as inter-vertex
communication is absolutely error-free. In other words, all nodes can be assumed to
always act in a fault-free manner, all sent messages can be assumed to never be lost
or undelivered. Therefore, we can always assume that messages sent and received by nodes
are all in accordence to the actual algorithm that is being executed by the nodes 
and not a result of an accidental fault or an intentional adversary effort.

\subsection{Distinctive qualities of LOCAL model}

As was already mentioned previously, our model of computation is known
as Linial's or \emph{LOCAL model}~\cite{Linial1987}. Therefore, to complete the
description, we will describe in more detail two distinctive qualities of LOCAL model
from other models of distributed computing, namely unique identifiers and arbitrarily large
bandwidth.

Each node in LOCAL model is provided with a unique identifier as part of initial input.
The identifiers are usually assumed to be integer numbers between 1 and $|V|^c$, where
$|V|$ is the number of nodes in the graph and $c$ is some constant. Unless specified, we
assume that such a constant $c$ is not known by the nodes at the beginning of
algorithm execution. Thus, unique identifiers are guaranteed to be positive integer
numbers bounded by a polynomial in the number of nodes, but it is not known - by the nodes
at the start of algorith execution - what is the largest unique identifier in the graph. Moreover,
the identifiers of the node do not necessarily form a continuous range of integers. That is
if an integer $x$ is used as a unique identifier for some node $v$, it is possible
that an integer $x+1$ is not used as an identifier in the graph. This implies that at
the beginning, nodes do not known what integers have been used for unique identifiers
and what have not been, with the exception of only one integer - their own identifier.

Another characteristic of LOCAL model, which also was already mentioned before, is the fact
that nodes can send (and receive) an arbitrarily large (but nevertheless finite)
amount of bytes of information over a single edge during one round. This fact,
combined with the existence of unique identifiers, renders the model as a rather strong one.
In particular, this implies that a node can send all of its information - no matter how
large it is - to all its neighbours in just one round.

This, in turn, implies a rather
curious property. Imagine that every node sends all of its information during the first round,
and having received some data from its neighbours, saves all this information locally. Consider
some node $v$ in the middle of a graph $G = (V, E)$. Since all nodes send all its data,
after the first round, node $v$
will have all the data that all its neighbours had initially, plus its own initial information.
Notice that each of $v$'s neighbours now have all the information of their neighbours. Thus,
if we combine all the data in posession of $v$ and $v$'s neighbours after the first round, it
is easy to observe that they together have all the information of $v$'s radius-2 neighbourhood.
But this means that after the second round, when all $v$'s neighbours have sent all their
information to $v$, $v$ alone posessses all the data that its radius-2 neighbourhood had at the
beginning of the algorithm execution. Similarly, after the 3rd round, $v$ will have all the
radius-3 neighbourhood's data, and so on.
In general, after round $x$, node $v$ will have
all information that was initailly available in its radius-$x$ neighbourhood.
Therefore, when considering LOCAl model,
time and space are in certain sense equivalent. In other words, the number of rounds
needed to solve a certain problem is always equal to the distance (or radius) to which a node needs to
see to solve a certain problem. One technicality to note here is that because all nodes have
unique identifiers, and these identifiers are sent together with the rest of data,
receiving nodes can differentiate what nodes the received data belongs to, and consequently,
reconstruct the structure of the neighbourhood in the graph.

As a consequence of the above,
we can observe that after $diam(G)$ rounds, node $v$ will have all the information there is in the 
graph $G$. This implies that after $diam(G)$ rounds, in LOCAL model, we can solve anything that can
be solved in a centralised setting. That is because at the end of $diam(G)$'th round, each node in the
graph have collected all the information there is in the graph and thus can just run the
computation locally. And because local computation in LOCAL model is virtually free,
anytihng that could be computed in a centralised setting will be computed locally by each
node individually. On the next round, each node can output their part of the solution.

\subsection{Randomised algorithms}

* intro to a randomised model

\section{Locally Checkable Labelling problems}

* Graph labelling problems

As the field developed, it became clear that some classes of problems
are of a particular intereset to the theoretical resdearch community.
One class of such problems has been first introduced in 1993 by
Moni Naor and Larry Stockmeyer under the name of Locally Checkable 
Labelling (LCL) problems.~\cite{Naor1993}.

// Exaplain LCLs in short here.asdasdasdasds

The study of LCL problems has been one of the major research directions
during the last 6-7 years, with numerous papers published in major
distributed computing conferences
~\cite{Balliu2016, Chang2016, Brandt2017, Chang2017, Fischer2017a, Rozhon2019, Balliu2020-1, Balliu2020-2}.

\section{Misc}

* log-star function

\section{Major LCL problems}

MM, MIS, vertex-coloring, edge coloring, etc.



\section{Recent developments in automated classification of LCL probems}





\section{Language and Structure}

Moreover, the transitions are also used in the paragraph and the
sentence level meaning that all the text is linked together. For example,
the word ``moreover'' here is one way, but of course you should use
variation in the text. Examples of transitional devices (words) and
their use can be found from writing guides, e.g. from the Academic
writing instructions of Aalto
University Language Center
\footnote{http://sana.aalto.fi/awe/ and especially for connecting words 
http://sana.aalto.fi/awe/cohesion/signposts/index.html/} of
Purdue University or Strunk's Elements of
Style\footnote{http://www.bartleby.com/141/}. Remember that footnotes
are additional information, and they are seldom used.  If you refer to a source, you do no
not use footnote. The right command for the references is \emph{cite},
and we will discuss about that later in this Chapter. 

Language Center of Aalto University offers many good courses for
thesis writes. For example, LC-1320 Thesis Writing for Engineers (MSc)
is planned to support writing the master's thesis 
and LC-1310 Academic Communication for MS Students covers both oral
and written language.

The language used in the thesis should be technical (or
scientifical). For example, the abreviations aren't used but all them
are written open (i.e. ``are not''). Since the content itself is often
hard to understand (and explain), the sentences should not be very
long, use complex language with several examples embedded in the same
sentence, and, also, seldom used words and weird euphemism or paraphrases
can make the sentence hard to follow and to read it with only one
time, and making everything even harder to understand all this without
any punctuation marks makes the instructor cry and finally after
trying to correct the language, you will get boomerang, and everyone's
time has just been wasted.

Please use proofreaders before sending even your unfinished version to
the instructor and/or supervisor. You will get better comments when
they do not need first proofread your text. Moreover, they can
consentrate to the content better if the language and spelling
mistakes are not distracting the reading. Several editors have their
own proofreading tools, e.g. ispell in emacs. You can also use
Microsoft Word to proofread your thesis: it can correct also some
grammatical errors and not just misspelled words. You can translate
your latex file to rtf with the \texttt{latex2rtf} command in the
kosh.aalto.fi shell server. Then, the line breaks
will not be problems for the proofreader of Word.

Note also that if you have a section or a subsection, you have to have
at least two of them, or otherwise the section or subsection title is
unnecessary. Same with the paragraphs: you should not have sections
with only one paragraphe, and single sentence paragraph. Furthermore,
always write some text after the title before the next level title.

\section{Finding and referring to sources}

Never ever copy anything into your theses from somebody else's text
(nor your own previously published text). Never. Not even for starting
point to be rewritten later. The risk is that you forgot the copied
text to your thesis and end up to be accused of plagiarism. Plagiarism
is a serious crime in studies and science and can ruin your career
even its beginning. To repeate: never cut and paste text into your
thesis!

\subsection{Finding sources}

All work is based on someone else's work. You should find the relevant
sources of your field and choose the best of them. Also, you should
refer to the original source where a fact has been mentioned first
time. Remember source evaluation (criticism) with all sources you
find.

Good starting points for finding references in computer science are: 
\begin{itemize}
% You can use this command to set the items in the list closer to each other
% (ITEM SEParation, the vertical space between the list items) 
\setlength{\itemsep}{0pt}
\item Aalto library's Computer Science Guide: \url{http://libguides.aalto.fi/computer} 
(in English) and \url{http://libguides.aalto.fi/tietotekniikka} (in Finnish)
\item Finna Portal (Aalto Library): \url{https://aalto.finna.fi/?lng=en-gb} (in English) 
and \url{https://aalto.finna.fi/} (in Finnish)
\item ACM Digital library: \url{http://portal.acm.org/}
\item IEEExplore: \url{http://ieeexplore.ieee.org}
\item ScienceDirect: \url{http://www.sciencedirect.com/}
\item \ldots although Google Scholar (\url{http://scholar.google.com/}) will
find links to most of the articles from the abovementioned sources, if you
search from within the university network
\end{itemize}

Some of the publishers do not offer all the text of the articles
freely, but the library has agreed on the rights to use the whole
text. Thus, you should sometimes use computers in the domain of the
university in order to get the full text. Sometimes the Finna Portal
can also help getting the whole article instead of just the abstract.
The library has also a self-study guide to information retrieval.

Instead of normal Google, use Google Scholar
(\url{http://scholar.google.fi/}). It finds academic publications whereas
normal Google find too much commercial advertisements or otherwise
biased information. Wikipedia articles should be referred to in the master
thesis only very, very seldomly. You can use Wikipedia for understanding
some basics and finding more sources, but often you cannot be sure if
the article is correct and unbiased.

One important part of the sources that you have found is the reference
list. This way you can find the original sources that all the other
research of the field refer. Often you can also find more information
with the name of the researchers that are often referred in the
articles.

\subsection{Sources and reference list}

The main point in referring to sources is to separate your own
thinking and text from that of others. Facts of the research area can
be given without reference, but otherwise you should refer to
sources. This means two things: marking the source in the text where
it has been used, and listing the sources usually in the end of the
thesis in a way that help the reader to find the original source. 
Aalto library has a comprehensive citation guide
.

There are several bibliography styles, meaning how to form the
bibliography in the end of the thesis and how to mark the references
in the text. You should ask from your supervisor or instructors which
style you should use. This thesis template uses the number style that
is often used in software engineering. Here, the bibliography is in
the alphabetical order, not in the order where the sources are
referred, and the sources are marked with numbers in the text. In all
styles, the key idea is to collect as much information of the sources
as is possible in the bibliography, and then let the latex environment take
care of organizing the necessary information to the reference list.

The other bibliographic styles are also used in the CS field. For example, usability
uses the Harvard style where instead of numbers, the reference is
marked into the text with author's name and publishing year. You can
change the bibliography style in the thesis-example.tex file. You get
the normal text reference, e.g. (Haapasalo, 2010), with latex command
\texttt{citet} or the plain \texttt{cite}, and with command
\texttt{citep}, you get the text reference ``Haapasalo (2010)'' that
you can use as subject of a sentence. Next, we tell more about how to mark
the references in the text.

\subsection{Referring to sources}

In addition to the list in the end of the thesis, you have to mark the
source in the text where the source is used. There are three places
for the reference: in a sentence before the period, in the end of a
sentence after the period, or in the end of a paragraph. All of them
have different meaning. The main point is that first you paraphrase
the source using your own words and then mark the source. Next, we
give short examples that are marked with \emph{emphasised text}.

\emph{Haapasalo researched database algorithms
  that allows use of previous versions of the content stored in the
  database.} This kind of marking means that this paragraph (or until
the next reference is given) is based on the source mentioned in the
beginning.  Giving the source you should use only the family name of
the first author of the article, and not give any hints about what is
the type of the article that is referred nor its title.

\emph{B+-trees offers one way to index data that is stored in to a
  database. Multiversion B+-trees (MVBT) offer also a way to restore
  the data from previous versions of the database. Concurrent MVBT
  allows many simultaneous updates to the database that is was not
  possible with MVBT.} When the marking is
after the period, the reference is retrospective: all the paragraph
(or after previous reference marking) is based on the source given in
its end. If the content is very broad, you can start with saying
\emph{According to Haapasalo}, then continue referring the source with
several separate sentences, and in the end put the marking of your
source \emph{ that shows that CMVBT are the
  best. }. 

If your paragraph has several sources, the above mentioned styles are
not proper. The reader of your thesis cannot know which of your
sources give which of the statements. In this case, it is better to
use more finegraded refering where the reference markings that are
embedded in the sentences. For example, \emph{the multiversion B+-tree
  (MVBT) index of Becker et al. allows database
  users to query old versions of the database, but the index is not
  transactional.
  It's successor, the transactional MBVT (TMVBT), allows a single transaction
  running in its own thread or process to update the database concurrently
  with other transactions that only read the
  database. 
  Further development, titled the concurrent MBVT (CMVBT),
  allows several transactions to perform updates to the database at the same
  time}. 
  Here, the references are marked before
  the period in the sentences where they are used. You should never
  but all these sources in the end of the paragraph. Referring several
  source at once should only used when you give a set of examples.

Finally, direct quotes are allowed. However, often you should avoid
them since they do not usually fit in to your text very well. Using
direct quotes has two tricks: quotation marks and the source.  \emph{
  ``Even though deletions in a multiversion index must not physically
  delete the history of the data items, queries and range scans can
  become more efficient, if the leaf pages of the index structure are
  merged to retain optimality.''} Quotes are
hard to make neatly since you should use only as much as needed
without changing the text. Moreover, you often do not really
understand what the author has mentioned with his wordings if you
cannot write the same with your own words. Remember also that never
cut and paste anything without marking the quotation marks right away,
and in general, never cut and paste anything at all!

Sometimes getting the original source can be almost impossible. In an
extremely desperate situation, you can refer with structure \emph{ms
  X~[\ldots] according to mr Y~[\ldots] defined that}, if you find a
source that refers to the original source. Note also that the
reference marking is never used as sentence element (example of how 
\textbf{not} to do it: \emph{ describes
an optimal algorithm for indexing multiversiond databases.}).



















