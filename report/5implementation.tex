\chapter{Implementation}
\label{chapter:implementation}

% \section{General architecture of the tool}

% \section{Problem representation}

% \subsection{Problem normalization}

% \section{Query representation}

% \section{Batch classification and reclassification}

% \section{Integration of Round Eliminator}

\section{Web interface}

This section will briefly discuss some implementation details of
the client side of the application. First, we start explaining why
the web interface has been implmented in the first place and why it
is important. Then, we describe some of the technologes that were
chosen as part of the implementation.

The reason why we decided to spend additional time and implement
the web interface can be best unnderstood when considering
another tool that has recently been rising in popularity among
the distributed algorithms community -- Round Eliminator~\cite{Olivetti2020}. The tool has proved to be a very useful utility
when doing research connected to LCL problems on biregular trees.
However, if it wasn't for the web interface that Olivetti has
implemented, it is most likely that the popularity and
frequency of use of the tool would have been nowhere near the current
levels. Indeed, almost all users of the tool use it via the web
client. Otherwise, a user would need to install Rust programming
language~\cite{FIXME}, build the tool locally using the tool
called "Cargo"~\cite{FIXME} and then run it using the command-line.
It is clear that the number of people willing to do that would be
significantly smaller than number of people who ended up using
an easy-to-use web interface that requires no installation or 
configuration.

By analogy, we can assume with high confidence that significantly more
user would be willing to use our tool vie the web interface rather
than downloading the soource of the tool locally and installing
a specific version of Python programming language~\cite{FIXME}.
Moreover, the richness of the web user interface allows us to convey complicated
ideas related distributed computing in way that is easier to understand, at least for knowledgable audience. Thus, it is has been
decided to spend some time resources -- even though they were 
quite limited -- on implementing a web user interface that is
relatively easy to use and understand compared to its
command line -based analog.


